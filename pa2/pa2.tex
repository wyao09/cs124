\documentclass[12pt]{article}

%AMS-TeX packages
\usepackage{amssymb,amsmath,amsthm} 
%geometry (sets margin) and other useful packages
\usepackage[margin=1.25in]{geometry}
\usepackage{graphicx,ctable,booktabs}


%Redefining sections as problems
%
\makeatletter
\newenvironment{problem}{\@startsection
       {section}
       {1}
       {-.2em}
       {-3.5ex plus -1ex minus -.2ex}
       {2.3ex plus .2ex}
       {\pagebreak[3]%forces pagebreak when space is small; use \eject for better results
       \large\bf\noindent{Problem }
       }
       }
       {%\vspace{1ex}\begin{center} \rule{0.3\linewidth}{.3pt}\end{center}}
       \begin{center}\large\bf \ldots\ldots\ldots\end{center}}
\makeatother


%
%Fancy-header package to modify header/page numbering 
%
\usepackage{fancyhdr}
\pagestyle{fancy}
%\addtolength{\headwidth}{\marginparsep} %these change header-rule width
%\addtolength{\headwidth}{\marginparwidth}
\lhead{Problem \thesection}
\chead{} 
\rhead{\thepage} 
\lfoot{\small\scshape CS124} 
\cfoot{} 
\rfoot{\footnotesize PA 1 Writeup} 
\renewcommand{\headrulewidth}{.3pt} 
\renewcommand{\footrulewidth}{.3pt}
\setlength\voffset{-0.25in}
\setlength\textheight{648pt}

%1%%%%%%%%%%%%%%%%%%%%%%%%%%%%%%%%%%%%%%%%%%%%%%

%
%Contents of problem set
%    

\title{CS 124 - Programming Assignment 2}
\author{Aidan Daly and Willie Yao}
\usepackage{fullpage}
\usepackage{graphicx}
\usepackage{epstopdf}

\begin{document}
\maketitle

\begin{problem}{}
\textbf{Implementation details and optimizations}\\
\\
We decided to represent our matrices as two-dimensional arrays of integers, and wrote helper functions for allocating and freeing them. \\
Random matrices for all tests were created such that each entry was uniformly random between 0, 1, and 2, though we also have support for testing under several other random integer schemes.\\
\\
\textbf{Conventional algorithm}

We decided to improve upon the efficiency of the conventional algorithm by using a more cache-friendly approach.  For a matrix multiplication of $A \times B$, the most naive method would iterate through the rows of $A$ and the columns of $B$, taking the dot product of each pair of vectors.  This approach would cause a lot of cache misses, however, as two-dimensional arrays in C are stored in row-major order.  This would mean $A$ would have no problem iterating through any row of the array, as the whole row will be loaded into the cache after the first miss, while iterating through a column of $B$ would result in a cache miss and a call to memory on every iteration.

To make our implementation more efficient, we wrote a helper function that would take a pointer to a square matrix and transpose it.  Thus, when multiplying $A \times B$, we would transpose $B$, iterate through $A$ and $B$ in row-major order, and dot product each pair of rows.  This minimized caches misses and greatly improved performance.
\begin{center}
\includegraphics[scale=.65]{figs/conventional.png}
\end{center}
\textbf{Strassen's algorithm}

Because Strassen's algorithm must operate on even matrices, at every level of recursion, it will only run all the way to a base case on square matrixes of a dimension that is some power of 2.  Thus, to account for matrices that are not a power of two, we had two options: 1) pad the matrices with zeroes up front such that their dimension will reach a power of two, or 2) pad up to an even dimension whenever an odd one comes up in the recursion.  

We opted for up-front padding because it would call for the least memory copying at the expense of a worst case of nearly doubling the dimension of a matrix up front (if its dimension is slightly larger than a power of two).

\end{problem}{}

\begin{problem}{}
Analytical work\\

\end{problem}{}
\begin{problem}{}
Experimental work\\

To experimentally determine the optimum cutoff value for Strassen's algorithm, we decided to run the algorithm on random matrices of dimension 4097 and compare the runtimes with different cutoff values.  We chose the problem dimension as 4097 because it is a relatively large number and also is one greater than a power of two.  This gives us a "worst case" runtime scenario for Strassen's algorithm, since it must pad this matrix to the next closest power of two - 8192.

After experimenting with a wide range of cutoff values (from 100 to 4000 in increments of 100), we obtained the graph below:
\begin{center}
\includegraphics[scale=0.8]{figs/benchmark-1.png}
\end{center}

This allowed us to refine our guess to be somewhere in the 100 to 1000 range.  Because our implementation of Strassen's algorithm only operates on matrices whose dimensions are powers of 2, we extensively tested the algorithm against 4097x4097 random matrix multiplications with cutoffs of $\{64, 128, 256, 512, 1024\}$.  The results for these tests are shown below:
\begin{center}
\includegraphics[scale=0.75]{figs/benchmark-2.png}
\end{center}

Thus, through our experimental work we have determined that the optimal cutoff for out implementation of Strassen's algorithm to switch over to the transpose-conventional implementation is 256.
\end{problem}{}

\end{document}